% !TeX spellcheck = pt_PT
\documentclass[12pt,a4paper]{article}
\usepackage[portuguese]{babel}
\usepackage[utf8]{inputenc}
\usepackage{natbib}
\author{José Pedro Sousa}
\title{Computaçao Paralela parte1}
\begin{document}

%pagina de rosto
\begin{titlepage}
	\centering
	{\scshape\LARGE Faculdade de Ciências \par}
	\vspace{1cm}
	{\scshape\Large Computação Paralela\par}
	\vspace{1.5cm}
	{\huge\bfseries All-Pairs Shortest Path Problem\par}
	\vspace{2cm}
	{\Large\itshape José Pedro Sousa\par}
	\vfill

	{\large \today\par}
\end{titlepage}
\tableofcontents
\section{Introdução}
A programação em paralelo leva-nos a uma abordagem diferente, quando comparada com linguagens mais sequenciais.

%A programação em lógica leva-nos a uma abordagem diferente, quando comparada com linguagens mais imperativas. Por norma, numa linguagem imperativa é necessário pensar passo por passo, como resolver o problema, já em programação lógica, não é necessário ter essa abordagem mas sim conseguirmos definir todos os factos e regras que se relacionam com o problema. A própria linguagem efetuará uma busca exaustiva até que encontre uma resposta que satisfaça a questão efetuada\cite{pires2013prolog}.


%Neste relatório iremos abordar duas linguagens como representantes de paradigmas diferentes, \textit{Prolog} e \textit{Python}. Prolog como linguagem de paradigma lógico e Python com uma linguagem imperativa.




\end{document}
